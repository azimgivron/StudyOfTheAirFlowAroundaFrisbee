\documentclass[10pt,a4paper]{report}
\usepackage[utf8]{inputenc}
\usepackage[francais]{babel}
\usepackage[T1]{fontenc}
\usepackage{amsmath}
\usepackage{amsfonts}
\usepackage{amssymb}
\usepackage{listings} % code highlights
\usepackage{fancyhdr} % headers and footers
\usepackage{graphicx}
\usepackage[left=2cm,right=2cm,top=2cm,bottom=2cm]{geometry}
\usepackage[nottoc]{tocbibind}


\author{Anuta Christian,Givron Azim}
\title{The frisbee's aerodynamism}


% Hide chapters numbers 
\renewcommand{\thesection}{\arabic{section}}


% Set headers and footers
\pagestyle{fancy}
\fancyhf{}
\lfoot{MECA-H3001: fluid mechanics and transfer process}
\cfoot{\thepage}
\renewcommand{\headrulewidth}{0pt}   % head horizontal rule 
\renewcommand{\footrulewidth}{0.5pt} % foot horizontal rule



\begin{document}



\begin{titlepage}

\includegraphics[scale=0.5]{logo-polytech-ULB-FR.jpg}

\center 
\vspace{5cm}
\textsc{\large MECA-H3001} \\[0.5cm]
\textsc{\LARGE Fluid mechanics and transfer process} \\[1.5cm]
\textsc{\Large English report} %\\[1.5cm]

\rule{\textwidth}{1pt}

\vspace{2cm}

\textsc{\large Anuta Christian, Givron Azim}

\end{titlepage}



\tableofcontents
\newpage 
\section{Introduction}
The frisbee is one of the objects that was invented decades ago and which is still used today. It is a source of amusement for kids and grown-ups, but also leads to new sports such as the ultimate frisbee or the disc golf. The particular shape of the frisbee allows it to glide for really long distances but to make the most of this shape, the toss has to be well oriented. In order to find the optimal angle, we need to use two basic’s physical concepts. The first one is the gyroscopic effect which tends towards to stabilize the flight and is simply due to the spinning momentum transmitted through the frisbee’s toss. The second one is the aerodynamic lift which allows the frisbee to fly and is caused by its movement in the air.
\section{Method}
The method used in order to find the optimal throw is trought computing simulations. In order to do that, the simulation must know the totallity of the forces, more precisely the aerodynamic forces. The two main aerodynamic forces acting on a frisbee are the drag and lift forces. To determine the magnitude of these forces two common physical relationships are used.

\section{Method}
The method used in order to find the optimal throw is trough computing simulations. In order to do that, the simulation must know the totality of the forces, more precisely the aerodynamic forces. The two main aerodynamic forces acting on a Frisbee are the drag and lift forces. To determine the magnitude of these forces two common physical relationships are used.

\subsection{The drag force}
To calculate the drag force, the Reynolds number, $\Re$, is required to know which drag relationship to apply. $\Re$ is given by,
\[\Re = \frac{\rho v d}{\eta}\]
where $\rho$ is the density of the fluid (in our case the air), v is the velocity of the fluid (or the velocity of the Frisbee relative to the fluid), d is the characteristic dimension of the object (for a Frisbee, the characteristic dimension is it’s diameter), and $\eta$ is the viscosity of the fluid. For a standard Frisbee thrown at sea level, the density of air is approximately 1.23 $kg/m^3$, the velocity of an average Frisbee throw is 14 m/s. The diameter of a standard Frisbee is 0.260 m and the viscosity of air is 1.73x$10^{-5}$ $(N.sec)/m^2$. This gives $\Re$ equal to 2.59x$10^5$. For a Reynolds number of this magnitude, the Prandtl relationship is needed to calculate the drag force, $F_d$ and it is given by,
\[F_d = - \frac{C_D \rho \pi \left(\frac{d}{2}\right)^2 v^2}{2}\]
in which the drag coefficient $C_D$ is present. This coefficient follows an empirical law\cite{art1} that varies with the angle of attack $\alpha$,
\[C_D = C_{D0} + C_{D\alpha}(\alpha-\alpha_0)^2\]
where all the coefficients (except $\alpha$) are found by experiment and depend on the physical aspects of the Frisbee. The Bernoulli Principle is a well known principle that states that there is a relationship between the velocity, pressure and height of a fluid at any point on the same stream line.

\subsection{The lift force}

The lift force applied on a Frisbee is very similar to the lift force on
airplane wings and is calculated using the Bernoulli principle.

\begin{thebibliography}{9}

\bibitem{art1}
  Hummel, Sarah A.,
  “Frisbee Flight Simulation and Throw Biomechanics”,
  University of Missouri,
  2003.

\end{thebibliography}
\end{document}