\documentclass[10pt,a4paper]{report}
\usepackage[utf8]{inputenc}
\usepackage[francais]{babel}
\usepackage[T1]{fontenc}
\usepackage{amsmath}
\usepackage{amsfonts}
\usepackage{amssymb}
\usepackage{listings} % code highlights
\usepackage{fancyhdr} % headers and footers
\usepackage{graphicx}
\usepackage[left=2cm,right=2cm,top=2cm,bottom=2cm]{geometry}
\usepackage[nottoc]{tocbibind}


\author{Anuta Christian,Givron Azim}
\title{The frisbee's aerodynamism}


% Hide chapters numbers 
\renewcommand{\thesection}{\arabic{section}}


% Set headers and footers
\pagestyle{fancy}
\fancyhf{}
\lfoot{MECA-H3001: fluid mechanics and transfer process}
\cfoot{\thepage}
\renewcommand{\headrulewidth}{0pt}   % head horizontal rule 
\renewcommand{\footrulewidth}{0.5pt} % foot horizontal rule



\begin{document}



\begin{titlepage}

\includegraphics[scale=0.5]{logo-polytech-ULB-FR.jpg}

\center 
\vspace{5cm}
\textsc{\large MECA-H3001} \\[0.5cm]
\textsc{\LARGE Fluid mechanics and transfer process} \\[1.5cm]
\textsc{\Large English report} %\\[1.5cm]

\rule{\textwidth}{1pt}

\vspace{2cm}

\textsc{\large Anuta Christian, Givron Azim}

\end{titlepage}



\tableofcontents
\newpage 
\section{Introduction}
The frisbee is one of the objects that was invented decades ago and which is still used today. It is a source of amusement for kids and grown-ups, but also leads to new sports such as the ultimate frisbee or the disc golf. The particular shape of the frisbee allows it to glide for really long distances but to make the most of this shape, the toss has to be well oriented. In order to find the optimal angle, we need to use two basic’s physical concepts. The first one is the gyroscopic effect which tends towards to stabilize the flight and is simply due to the spinning momentum transmitted through the frisbee’s toss. The second one is the aerodynamic lift which allows the frisbee to fly and is caused by its motion through the air and the pressure acting on it.

\section{Method}
In order to find the optimal throw, we simulated the frisbee's trajectory by the computing simulations. This requiers to know the main forces acting on it which are the gravitational force and the lift and drag forces. The latter is the force exerts by the air on the frisbee in the flow direction while the lift is the normal force applied on the frisbee. Since those forces acte only through the normal and parallel direction of the frisbee, we simplified the simulation by studying only the trajectory in these tow dimension.

TO DO FIND ILLUSTRATION

\subsection{Calculation of the forces}
The three forces cited above are all expressed by physical relationships which allowed us to fine them. in the following part, we will present these relationships.
\\
The gravitational force $F_g$ is proportional to the mass $m$
\[F_g = m g\]
with $g$, the gravitational acceleration.
\\
The resolution of the equations to obtain the drag and lifts forces can be very difficult to solve because they depend on the geometry of the object studied. Fortunately, we can approximate the frisbee's geometry to a flat plate moving alongside the flow. In this case, the drag and lift forces are respectively given by :
\[F_d = -\frac{C_d \rho A  v^2}{2}\]
\[F_l = \frac{C_l \rho A  v^2}{2}\]
In this particular case, the fluid is the air, thus $\rho$ is its density, $v$ is the velocity of the Frisbee relative to the air, A is the surface area of the plate exposed to the air flow and $C_d$ and $C_l$ are respectively the drag and lift coefficients.
The drag coefficient, in general, depends on the Reynolds number which gives the type of flow studying. A flow can be laminar or turbulent. The first one concerns ordered flows while the second one caracterize choatics flows. This number can be calculated with the following relationship:
\[\Re = \frac{\rho v d}{\eta}\]
where the only new term appearing here is $\eta$, the viscosity of the air.
By using the data in annexes, we found $Re=2.59$ $10^5$ which correspond to a turbulent flow.
This leads us to another equation:
\[C_d = C_{d0} + C_{d\alpha}(\alpha-\alpha_0)^2\]
which links the drag coefficient $C_d$ to the angle of attack $\alpha$ and the other terms $C_{d0}$, $C_{d\alpha}$ and $\alpha_0$ are constants, found experimentally.
\\Finally, the lift coefficient can be found with the relationship below:
\[C_l = C_{l0} + C_{l \alpha} \alpha\]
where $C_{l0}$ and $C_{l\alpha}$ are also constant found experimentally.


\begin{thebibliography}{9}

\bibitem{art1}
  Hummel, Sarah A.,
  “Frisbee Flight Simulation and Throw Biomechanics”,
  University of Missouri,
  2003.
  
 \bibitem{art2}
  Lissaman P., Hubbard M.,
  “Maximum range of flying discs”,
  University of California,
  2010.
  
  \bibitem{art3}
  Morrison V. R.,
  “The Physics of Frisbees”,
  Mount Allison University,
  2005.

\bibitem{art4}
  Koyanagi R., Seob K., Ohta K., Ohgi Y.,
  “A computer simulation of the flying disc based on the wind tunnel test data”,
  University of Keio \& Yamagata,
  2012.
  
  \bibitem{art5}
  Baumback, Kathleen,
  “The Aerodynamics of Frisbee Flight”,
  University of South Florida,
  2010.
  
  \bibitem{art6}
  Erynn J. Schroeder,
  “An Aerodynamic Simulation of Disc Flight”,
  College of Saint Benedict/Saint John's University,
  2015.

\end{thebibliography}
\end{document}